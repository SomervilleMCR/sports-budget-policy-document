\input{"input/Preamble & Start"}

\organisation{Somerville College\\MCR Committee}
\title{Sports Budget Policy Document}
\subtitle{Version 5}
\author{Rowan Nicholls}
\date{\today}

\input{"input/Heading"}

\section*{Preamble}
The MCR Committee considers sport to be a valuable and beneficial facet of life - physically, mentally and socially. In addition to the positive effects for the individuals involved, there are also broader opportunities to foster team spirit, friendship and community within the MCR, to create deeper connections with alumni and to improve the image of Somerville within Oxford and elsewhere. The Committee thus commits itself to supporting MCR members who take part in sport by setting aside a Sports Budget that can be applied to. In practice, however, the Committee has limited resources and so may not be able to completely support everyone who applies. There is thus a need to assess and scrutinise applications - on an ad hoc basis - with priority being given to financial claims for specific items related to Somerville representation at College level, to first-team representation at University level and/or to higher achievements not related to Oxford.

Note that this is a policy document and is thus not intended to be voted upon, ratified or formally adopted by the Committee. Its purpose is merely to verbalise the intended procedure and to serve as a guideline.

\section{Details}
Information about the Sports Budget and the process to apply to claim from it are introduced here.
\begin{itemize}
	\item The MCR Committee will allocate funds to a Sports Budget
	\item This Budget will be administered by the Sports Rep
	\begin{itemize}
		\item If the MCR has not elected a Sports Rep it should be administered by the Treasurer by default, or by another Committee member if it is so decided. This document will use the term `Sports Rep' to generically mean `the person administering this Budget'.
		\item The Sports Rep is a non-executive position and, as such, this role does not carry with it the right to vote in Committee decisions nor to approve or make budget changes or payments. The role is rather to collate the details of the applications to the Budget and present them to the Committee at meetings for approval. The Sports Rep can, of course, offer their opinions on the applications and speak in Committee meetings on these or any other topics, as is the right of all MCR members.
	\end{itemize}
	\item MCR members may apply to receive money from the Budget by contacting the relevant person.
	\begin{itemize}
		\item Both Full and Associate MCR members may submit applications, although the Committee reserves the right to prioritise Full members over Associate members
		\item Only current members (Full or Associate) may apply
		\item Applications on behalf of another member or on behalf of an MCR team will be accepted
		\item Members of the Committee may apply, including the Sports Rep themself.
	\end{itemize}
	\item There is no official application form. As long as the Committee is satisfied that they have enough details about what the application is for and how much is being asked for - and the Sports Rep has enough information to be satisfied that the person is a bona fide MCR member and sufficient proof-of-payment is provided - the application can be considered.
	\item At each Committee meeting, the Sports Rep will bring all applications received since the last meeting before the Committee.
	\begin{itemize}
		\item The Sports Rep should be unbiased when presenting the applications to the Committee. However, it is acknowledged that they themselves have a right to have an opinion on whether the application should be accepted or not so this requirement should be interpreted reasonably.
		\item It is good practice to present the applications anonymously, although it it not necessary (often the identity of the applicant is obvious from context anyway).
		\item If the applicant is a committee member there is no need for them to leave the room or recuse themselves from the discussion. This includes if the applicant is the Sports Rep themself.
		\item If the Sports Rep is not going to be present at the meeting, they should forward the applications on to other member(s) of the committee so that they can be brought up and discussed.
		\item If an application is received out of term time it will normally only be discussed at the next meeting, which may be a significant amount of time after the application is received. 
		\item If the Sports Rep feels that a decision should be reached on an application sooner than waiting for the next meeting would allow - eg because it is urgent or because it is out of term time and thus the next meeting is a long way off - it can be sent to the Executive Committee members via email and voted upon via reply
	\end{itemize}
	\item The Executive Committee members should endeavour to reach a decision on each application in the same meeting in which it is brought up
	\item The Committee may choose to approve a claim in full, in part or not at all. If the claim is not approved in full, justification should be provided to the applicant.
	\item The amount paid will be taken out of the Committee's Sports Budget for the current period
	\begin{itemize}
		\item If there is ambiguity - eg if the application was approved out of term time - the Committee may make an ad-hoc decision as to how to best represent the expense relative to the Budget
		\item Similarly, if there is a limit imposed as to how much a single person may be awarded in a given time period and there is ambiguity as to when this period starts and ends, an ad-hoc decision can be made
	\end{itemize}
	\item In years where the Committee has a surplus of funds they made decide to combine the Sports Budget with an Arts Budget. This can be considered as two budgets drawing from the same pool of money. While there is no Arts Budget Policy Document, this Sports Budget document can be used to inform discussion about whether or not an application to the Arts Budget should be approved.
\end{itemize}

\section{What Can and Cannot Be Claimed For}
Applications that fall into the following categories will be considered (this list is not definitive and should be taken as a guideline):
\begin{itemize}
	\item Expenses related to competing for an MCR, College or University team, or at a higher level
	\begin{itemize}
		\item This also includes individual events where one represents the college, eg Athletics Cuppers
		\item It is not necessary to be in the first-team, although the Committee may take the level of competition into account.
		\item Specific items with definite, known costs are preferable:
		\begin{itemize}
			\item Entry fees to a specific competitive event
			\item Transport to competitive events
			\item Equipment and/or kit that is necessary for taking part in a competitive event\end{itemize}
		\item Items that will be used in competition, as opposed to in practice or training, will be considered:
		\begin{itemize}
			\item Match kit will be considered while practice kit/`stash' will not. In general, only that which is necessary to officially represent a team that someone will definitely be representing will be considered.
		\end{itemize}
	\end{itemize}
	\item Expenses involved in organising MCR social sports events or sports-related events which are advertised to all MCR members
	\begin{itemize}
		\item In the same way that the MCR social secretaries organise social events that are open to all MCR members, the Sports Rep and/or others may organise social events that are sporting in nature. Some past examples are:
		\begin{itemize}
			\item Football kickabouts
			\item Sports days
			\item Trips to watch Oxford United
			\item Poker nights
			\item Beer pong
		\end{itemize}
		The organisers are reimbursed in the same way that social secretaries are.
		\item General sports equipment bought for the MCR (eg frisbees, balls, foam rollers, therabands, cones, etc) count
		\item In general, anything that serves as an activity associated with the MCR that any MCR member can get involved in is encouraged. An example is organising a group of MCR members to enter the Town \& Gown 10k race or the Teddy Hall Relays together as an MCR team.
	\end{itemize}
	\item Blues blazers
	\begin{itemize}
		\item College used to have a central fund to help students buy Blues blazers but this was devolved to the JCR and MCR Committees. As a result, these are now explicitly claimable from the Sports Budget.
	\end{itemize}
	\item As per the Constitution, the Committee cannot unilaterally approve expenses of \pounds 200 or more and must instead seek approval at a General Meeting for these. The substance of this Policy Document should apply when discussing applications at a General Meeting as well.
	\item If an event was cancelled for reasons beyond the applicant's control, non-recoverable expenses related to the event (eg if they had already travelled to the event before it was cancelled) will still be considered. A similar policy applies to events that a person misses due to genuine reasons such as illness.
\end{itemize}

Applications that fall into the following categories will \textit{not} be considered (this list is also not definitive and should also be taken as a guideline):
\begin{itemize}
	\item Costs that were incurred by a club or team as a whole as opposed to by an individual\begin{itemize}
		\item For example, if a sports club pays to enter a team into a competition and, additionally, each team member must pay for his/her own transport to and from the competition, only the latter can be claimed for
		\item An exception to this is MCR sports teams, eg an MCR football team, where fees such as Cuppers entry and team expenses such as goalkeeping gloves are accepted.
	\end{itemize}
	\item Practice kit, `stash', items not necessary for actual competition
	\item Subscription fees (`subs'), membership fees, affiliation fees, etc
	\begin{itemize}
		\item This is because subs represent the cost that must be paid before someone even starts participating, and the Committee feels that this cost for the opportunity to participate must be borne by the individual
		\item Additionally, just because someone has joined a club does not necessarily mean that they are participating or competing for it, and the Committee has no practical way to verify that someone is genuinely participating in a sport
		\item This is why, in general, items related to bone fide competition will be reimbursed but not items related to membership, practice or training.
		\item However, situations in which subscription fees are particularly high or proving to be an obstacle to participation will indeed be considered.
	\end{itemize}
	\item Embellishments such as shirt embroidery and/or printing on kit
	\item Gym membership
	\item Competing for department/faculty sports teams
	\item Entry into events as an individual, eg the Oxford Half Marathon
	\item If an event is missed for avoidable reasons reimbursement will not be considered
\end{itemize}

A precedent document that lists past applications and their outcomes can be found \href{https://docs.google.com/document/d/1C7BfB_jgS3wMEleUC6G4ZU182yfILvuTf4rHGR9wSyk/edit}{\textit{here}}

\section{Additional Factors}
The Committee should consider the following when reviewing an application:
\begin{itemize}
	\item Amount of money left in the Budget relative to the amount of time left in the year/number of expected applications that will still be received
	\item Selectiveness of the sport/the event. Mob events (ones that anyone can enter without needing to be selected) will be less likely to be reimbursed if the Budget is running low.
	\item The amount that the applicant has claimed so far that year - the Budget should ideally be spread amongst many people rather than a few
	\item If the person would be unable to take part if funding was not provided
	\item Reasonable costing
	\begin{itemize}
		\item For example, if someone buys an extremely expensive pair of football boots to play matches in, the Committee reserves the right to look up the price of an `average' pair of boots and only consider reimbursing that amount
	\end{itemize}
	\item While the Committee is not obliged to spend its entire Budget or approve claims it does not consider worthy, it should be noted that the money in the Budget has been earmarked to support MCR members taking part in sport. Thus, if it is apparent that the year will run considerably under-budget it is justifiable to relax the conditions for approval as much as is deemed fit.
	\item If the year ends with money still left in the Budget, the Committee may choose to retroactively review the applications that were received during that year and make additional payments. However, the Committee reserves the right to not make any additional payments, not perform retroactive reviews and not completely exhaust that year's Sport Budget should they choose.
\end{itemize}

The Committee should \textit{not} consider the following when reviewing an application:
\begin{itemize}
	\item Status of sport (eg if it is a Full-Blue vs a Half-Blues sport) or status of event (eg BUCs \& Varsity Matches for University teams vs Cuppers events for College teams)
\end{itemize}

\section{Maximum Reimbursements}
Usually, the Committee will set a maximum amount that can be awarded to a single person in a time period (usually a year). Often, this is equal to the maximum amount that can be awarded to someone who applies to the Barbara Craig Fund.

\section{Pre-Approvals}
Applications for future expenses will be considered.
\begin{itemize}
	\item The Committee has the option of pre-approving such applications (ie agreeing to pay the applicant only once the cost has been incurred, after being sent proof-of-payment)
	\item The advantage to this method of application is that the person can then immediately be transferred the money by the treasurer once they have incurred the cost instead of waiting for the next meeting.
\end{itemize}

\end{document}
