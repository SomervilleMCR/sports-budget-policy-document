\newcommand{\doctitle}	{Sports Budget Policy Document}
\newcommand{\docauthor}	{Rowan Nicholls}
\newcommand{\thedate}	{\today}
\newcommand{\comment}	{Version 4}

\input{"input/Preamble & Start"}
\input{"input/Heading"}

\section*{Preamble}
The MCR Committee considers sport to be a valuable and beneficial facet of life - physically, mentally and socially. In addition to the positive effects for the individuals involved, there are also broader opportunities to foster team spirit, friendship and community within the MCR, create deeper connections with alumni and improve the image of Somerville within Oxford and elsewhere. Practically, the Committee cannot completely support all MCR members who participate in sport and any level, and thus focus will be on financial claims for specific items related to Somerville representation at College level, first-team representation at University level and/or higher achievements not related to Oxford.

Not that this is a policy document and thus is not intended to be voted upon, ratified or formally adopted by the Committee. It is merely to verbalise the intended procedure and to serve as guidelines.

\section{Particulars}
Details of the Sports Budget and the process to apply to claim from it are introduced here.
\begin{itemize}
	\item The MCR Committee will allocate funds to a Sports Budget
	\item MCR members may apply to receive money from this budget by contacting the Sports Rep (or Secretary, if no Sports Rep has been appointed)
	\item Any type of member may submit an application (applications on behalf of another member or on behalf of an MCR team will also be accepted), although the Committee reserves the right to prioritise Full members over Associate members, for example
	\item An application must include:
	\begin{itemize}
		\item the applicant's name
		\item the applicant's Somerville email address (or the email address used to apply to the MCR if they do not have a Somerville one), this is to allow the Committee to confirm the applicant's membership status
		\item the relevant sports club
		\item the applicant's role in the sports club/team and/or the event concerned
		\item a specific expense they have or will incur as part of their involvement with that sports club
		\item the date on which the expense was or will be incurred (if unknown, the best estimate should be given)
		\item if the expense has not yet been incurred, the applicant should justify how they know the cost and why they will incur it (e.g.\ they have already been selected for the event and they know the entry fee that they will have to pay)
	\end{itemize}
	\item Members of the Committee may also apply
	\item At each meeting of the Committee, all applications received since the previous Committee Meeting will be discussed. If one or more members of the Committee have submitted an application themselves, they should recuse themselves and leave the room for this period.
	\item The discussions will be lead by the Sports Rep. If the Sports Rep is not going to be present at the meeting, the applications to be discussed should be forwarded to another member of the Committee who will then lead the discussion instead. This also applies if the Sports Rep has submitted an application him- or herself: they cannot lead the discussions in this instance because they would need to recuse themselves.
	\item If the discussion does not reach consensus, a vote may be forced by the Sports Rep/equivalent
	\item The Committee may choose to approve a claim in full, part or not at all. If the claim is not approved in full, justification must be provided to the applicant.
\end{itemize}

\section{What Can and Cannot Be Claimed For}
Applications to claim for the following categories will be considered (note that this list is not exclusive but should be taken as a guideline):
\begin{itemize}
	\item Specific items with known costs can be claimed for, for example:
	\begin{itemize}
		\item entry fees to a specific competitive event
		\item transport to competitive event(s)
		\item equipment and/or kit that will be used in competition - not practice kit
	\end{itemize}
	\item Only 'competition' items can be claimed for in order to prevent people from buying a large amount of stash and claiming for all of it. Only that which is necessary to officially represent the team should be claimed for
	\item Only costs that were incurred by an individual (who muct be an MCR member) can be claimed for, as opposed to costs incurred by a club or a team as a whole (e.g.\ if the rowing club pays to enter a crew into an external regatta and each member must pay for his/her own transport to and from the regatta, only the latter can be claimed for)
	\item If the application concerns a college-level sport club, all competitive events can be considered. These include Cuppers events and open competitions where a team enters under the Somerville name and represents the College (e.g.\ the Teddy Hall Relays)
	\item If the application concerns a university-level sport club, only first-team level competitive events can be considered (e.g.\ BUCS, first-team league and Varsity Matches)
	\item If the application concerns a sports club/team unrelated to Oxford, only competitive events that have a higher importance than university sport (e.g.\ national representation) can be considered
	\item Claims for reimbursements of subscription fees ('subs') are generally not accepted. As per the preamble, the Committee is not looking to support participation per se, but rather participation in a particular events/teams/etc only. However, situations in which subscription fees are particularly high or proving to be an obstacle to participation will indeed be considered.
	\item Gym membership is unlikely to be reimbursed. In generally, items related to competition will be reimbursed but not items related to practice/training.
\end{itemize}

\section{Additional Factors Considered when Reviewing Claims}
The Committee should consider the following when reviewing an application:
\begin{itemize}
	\item Amount of money left in the budget relative to the amount of time left in the year/number of expected applications that will still be received
	\item Status of sport. If there are a large number of applications relative to the amount of money available, the Committee may choose to approve claims relating to more prominent sports before smaller sports. The Oxford Sports Department's list of Full- and Half-Blues sports can be used in this regard.
	\item Status of event. Likewise, the Committe may choose to prioritise more prominent events over smaller ones (e.g.\ BUCs \& Varsity Matches for University teams, Cuppers events for College teams).
	\item Selectiveness of the sport/the event. Mob events (ones that anyone can enter without needing to be selected) will be less likely to be reimbursed.
	\item The amount that the applicant has claimed so far that year - the budget should ideally be spread amongst many people rather than a few
	\item If the person would be unable to take part if funding was not provided
	\item Reasonable costing. For example, if someone buys an extremely expensive pair of football boots to play matches in, the Committee reserves the right to look up the price of an 'average' pair of boots and only consider reimbursing that amount
	\item While the Committee is not obliged to spend its entire budget or approve claims it does not consider worthy, it should be noted that the money in the budget has been earmarked to support MCR members taking part in sport. Thus, if it is apparent that the year will run considerably under-budget, it is justifiable to relax the conditions for approval as much as is deemed fit.
	\item If the year ends with money still left in the budget, the Committee may choose to retroactively review the applications that were received during that year and make additional payments. However, the Committee reserves the right to not make any additional payments, not perform retroactive reviews and not completely exhaust that year's Sport Budget should they choose.
\end{itemize}

\end{document}